%!Tex Root = ../main.tex
% ./Packete.tex
% ./Design.tex
% ./Deklarationen.tex
% ./Vorbereitung.tex
% ./Aufgabe2.tex
% ./Aufgabe3.tex
% ./Aufgabe4.tex
% ./Appendix.tex

\section{Exercise 1}

\setcounter{exercise}{1}

% https://tex.stackexchange.com/questions/210450/print-counter-value-minus-1
\begin{frame}[allowframebreaks, fragile]{Exercise \thesection}{Directed Graphs}
  \begin{exercisenoinc}
    \begin{columns}
      \begin{column}{0.5\textwidth}
        \begin{itemize}
          \item Digraph $G=\left(V,E,\phi\right)$:
          \begin{align*}
            V &= \{v_0, v_1, ..., v_7\} \\
            E &= \{r_0, r_1, ..., r_{15} \} \\
            \phi(r_i) &= (v_{\lfloor i / 2\rfloor}, v_{i \; \text{mod} \; 4})
          \end{align*}
        \end{itemize}
      \end{column}
      \begin{column}{0.5\textwidth}
        \ctikzfig{1a_empty}
      \end{column}
    \end{columns}
  \end{exercisenoinc}
  \begin{requirementsnoinc}
    \centering
    \begin{minipage}{0.6\textwidth}
      \begin{linenums}
        \numberedcodebox[minted language=python, minted options={}]{./figures/edges.py}
      \end{linenums}
    \end{minipage}
    \begin{terminal}
      |\prompt|./edges.py
      [(0, 0), (0, 1), (1, 2), (1, 3), (2, 0), (2, 1), (3, 2), (3, 3), 
       (4, 0), (4, 1), (5, 2), (5, 3), (6, 0), (6, 1), (7, 2), (7, 3)]
    \end{terminal}
  \end{requirementsnoinc}
\end{frame}
\begin{frame}[allowframebreaks]{Exercise \thesection}{Directed Graphs}
  \begin{solution}
    \begin{columns}
      \begin{column}{0.5\textwidth}
        \begin{itemize}
          \item Digraph $G=\left(V,E,\phi\right)$:
          \begin{align*}
            V &= \{v_0, v_1, ..., v_7\} \\
            E &= \{r_0, r_1, ..., r_{15} \} \\
            \phi(r_i) &= (v_{\lfloor i / 2\rfloor}, v_{i \; \text{mod} \; 4})
          \end{align*}
        \end{itemize}
      \end{column}
      \begin{column}{0.5\textwidth}
        \ctikzfig{1a}
      \end{column}
    \end{columns}
  \end{solution}
  \begin{exercisenoinc}
    \begin{itemize}
      \item Is it simple?
      \begin{itemize}
        \item[]
      \end{itemize}
      \item Is it Eulerian?
      \begin{itemize}
        \item[]
      \end{itemize}
    \end{itemize}
  \end{exercisenoinc}
  \begin{solution}
    \begin{columns}
      \begin{column}{0.5\textwidth}
        \begin{itemize}
          \item Is it simple?
            \begin{itemize}
              \item The diagram is simple in terms of the definition of the lecture.
            \end{itemize}
          \item Is it Eulerian?
            \begin{itemize}
              \item Not Eulerian, for a digraph to be Eulerian we require that $d^{+}(v)\,=\,d^{-}(v)$ for all vertices. However this is not the case for $v_i, i > 3$
            \end{itemize}
        \end{itemize}
      \end{column}
      \begin{column}{0.5\textwidth}
        \ctikzfig{1a}
      \end{column}
    \end{columns}
  \end{solution}
  \begin{exercisenoinc}
    \begin{itemize}
      \item $v(0): d^{+}=\qquad ,d^{-}=\qquad ,N^{+}=\{\qquad, \qquad\},N^{-}=\{\qquad ,\qquad ,\qquad ,\qquad \}$
      \item $v(2): d^{+}=\qquad ,d^{-}=\qquad ,N^{+}=\{\qquad, \qquad\},N^{-}=\{\qquad ,\qquad ,\qquad ,\qquad \}$
      \item $v(7): d^{+}=\qquad ,d^{-}=\qquad ,N^{+}=\{\qquad, \qquad\},N^{-}=\{\qquad \}$
      \item $n(G)=|V(G)|=\qquad$ and $e(G)=|E(G)|=\qquad$
    \end{itemize}
  \end{exercisenoinc}
  \begin{solution}
    \begin{columns}
      \begin{column}{0.5\textwidth}
        \begin{itemize}
          \item $v(0): d^{+}=2,d^{-}=4,N^{+}=\{v_{0},v_{1}\},N^{-}=\{v_{0},v_{2},v_{4},v_{6}\}$
          \item $v(2): d^{+}=2,d^{-}=4,N^{+}=\{v_{0},v_{1}\},N^{-}=\{v_{1},v_{3},v_{5},v_{7}\}$
          \item $v(7): d^{+}=2,d^{-}=0,N^{+}=\{v_{2},v_{3}\},N^{-}=\{\}$
          \item $n(G)=|V(G)|=8$ and $e(G)=|E(G)|=16$
        \end{itemize}
      \end{column}
      \begin{column}{0.5\textwidth}
        \ctikzfig{1a}
      \end{column}
    \end{columns}
  \end{solution}
  \begin{exercisenoinc}
    \begin{itemize}
      \item For $d^+$ it is $\qquad$, for $d^-$ the first four, so $v_0, v_1, v_2, v_3$ holds $d^- = \qquad$ and for all other $\qquad$.
      \item So $\delta(G)^{+}=\qquad, \,\delta(G)^{-}=\qquad$ and $\Delta(G)^{+}=\qquad,\,\Delta(G)^{-}=\qquad.$
    \end{itemize}
  \end{exercisenoinc}
\end{frame}
\begin{frame}[allowframebreaks]{Exercise \thesection}{Directed Graphs}
  \begin{solution}
    \begin{columns}
      \begin{column}{0.5\textwidth}
        \begin{itemize}
          \item For $d^+$ it is $2$, for $d^-$ the first four, so $v_0, v_1, v_2, v_3$ holds $d^- = 4$ and for all other $0$.
          \item $\mathrm{So~}\delta(G)^{+}=2, \,\delta(G)^{-}=0\mathrm{~and~}\Delta(G)^{+}=2,\,\Delta(G)^{-}=4.$
        \end{itemize}
      \end{column}
      \begin{column}{0.5\textwidth}
        \ctikzfig{1a}
      \end{column}
    \end{columns}
  \end{solution}
\end{frame}
