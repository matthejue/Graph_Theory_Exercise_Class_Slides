%!Tex Root = ../main.tex
% ./Packete.tex
% ./Design.tex
% ./Deklarationen.tex
% ./Vorbereitung.tex
% ./Aufgabe1.tex
% ./Aufgabe2.tex
% ./Aufgabe3.tex
% ./Aufgabe4.tex

\section{Appendix}
% letztes Übugsblatt Aufgabe 3, generell nochmal mit Musterlösung vergleichen
% Struktur, Modell, Mengenrepräsentation, Begriffe
% Homomorphismen

\begin{frame}[allowframebreaks]{Appendix}{Sufficient and Necessary}
  \begin{itemize}
    \item \alert{Sufficient:} If $A\rightarrow B$ then $A$ is a sufficient condition for $B$
      \begin{itemize}
        \item for $\neg A$ it can be both $\neg A \rightarrow B$ or $\neg A \rightarrow \neg B$
      \end{itemize}
    \item \alert{Necessary:} If $A\leftarrow B$ then $A$ is a necessary condition for $B$, because of $\neg A \rightarrow \neg B$
      \begin{itemize}
        \item for $A$ it can be both $A \rightarrow B$ or $A \rightarrow \neg B$
      \end{itemize}
  \end{itemize}
  \begin{figure}
    \begin{subfigure}[t]{0.4\textwidth}
        \resizebox{\columnwidth}{!}{%
          \vbox{
            \ctikzfig{sufficient}
          }
        }
        \caption{$A \subset B$, sufficient}
    \end{subfigure}
    \begin{subfigure}[t]{0.4\textwidth}
      \resizebox{\columnwidth}{!}{%
        \vbox{
          \ctikzfig{necessary}
        }
      }
      \caption{$A \supset B$, necessary}
    \end{subfigure}
  \end{figure}
  \begin{table}
    \centering
    \begin{tblr}{
      cells = {BoxColor},
      row{1} = {PrimaryColor,fg=white},
    }
    $A$ & $B$ & $A\rightarrow B $ & $B\rightarrow A$ & $A\Leftrightarrow B$ \\
      0  &  0  &  1   &  1   &   1  \\
      0  &  1  &  1   &  0   &   0  \\
      1  &  0  &  0   &  1   &   0  \\
      1  &  1  &  1   &  1   &   1  
    \end{tblr}
  \end{table}
  \begin{Sidenote}
    \begin{itemize}
      \item if $A\rightarrow B$ then for $A$ we \alert{only} know for sure $\neg B\rightarrow \neg A$, for $\neg A$ it can \alert{both} be the case that $\neg A\rightarrow B$ or $\neg A\rightarrow \neg B$,
      \item thus for $A\leftrightarrow B$ we know $\neg A\leftrightarrow \neg B$, because we know $\neg B\rightarrow \neg A$ and $\neg A\rightarrow \neg B$ are both the case because $A\leftrightarrow B$ \alert{iff} $B\leftarrow A$ and $A\leftarrow B$
      \item in \alert{german}: \enquote{hinreichend und notwendig}
    \end{itemize}
  \end{Sidenote}
\end{frame}

\begin{frame}[allowframebreaks]{Appendix}{Languages and Structures}
  \begin{itemize}
    \item \alert{Language:} ${\mathcal{L}}=\{c_{i}\}_{i\in I}\cup\{f_{j}\}_{j\in J}\cup\{R_{k}\}_{k\in K}.$
    \item \alert{$\mathcal{L}$-Structure:} $\mathcal{A}=(A,\{c_{i}^{\mathcal{A}}\}_{i\in I},\{f_{j}^{\mathcal{A}}\}_{j\in J},\{R_{k}^{\mathcal{A}}\}_{k\in K}\}$ with:
    \begin{itemize}
      \item we have a non-empty \alert{universe} $A$ together with interpretations of the \alert{constant}, \alert{function} and \alert{relation} signs
      \item for every \alert{constant sign} $c_i$ there's a element from $A$ that we call $c_i^{\mathcal{A}}$
      \item for every \alert{function sign} $f_j$ with arity $n_j$ there's a function $f_j^{\mathcal{A}}: A^{n, j} \rightarrow A$
      \item for every \alert{relation sign} $R_k$ with arity $n_k$ there's a subset $R_k^{\mathcal{A}}$ of $A^{n, k}$
    \end{itemize}
  \end{itemize}
\end{frame}

\begin{frame}[allowframebreaks]{Appendix}{Graphs - Function or only Relation?}
  \begin{columns}
    \begin{column}{0.6\textwidth}
      \ctikzfig{graph_examples}
    \end{column}
    \begin{column}{0.4\textwidth}
      \begin{align*}
        G_3 =\;&\left(V,E,\phi\right):\\
        V =\;&\{v_0, v_1, v_2\}, \\
        E =\;&\{e_0, e_1, e_{2} \}, \\
        \phi = &\{e_0 \mapsto (v_0, v_1),\\ 
               &\enspace e_1 \mapsto  (v_1, v_2)\} 
      \end{align*}
    \end{column}
  \end{columns}
\end{frame}
