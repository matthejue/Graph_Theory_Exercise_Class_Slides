%!Tex Root = ../main.tex
% ./Packete.tex
% ./Design.tex
% ./Deklarationen.tex
% ./Vorbereitung.tex
% ./Aufgabe1.tex
% ./Aufgabe2.tex
% ./Aufgabe3.tex
% ./Aufgabe4.tex

\section{Appendix}
% letztes Übugsblatt Aufgabe 3, generell nochmal mit Musterlösung vergleichen
% Homomorphismen, Begriff Mengenrepräsentation
% Struktur, Modell

\begin{frame}{Appendix}{Search algorithms}
  \resizebox{\textwidth}{!}{
    \begin{minipage}[t]{120cm}
      % https://tex.stackexchange.com/questions/636543/how-can-i-center-and-scale-a-mindmap
      \centering
      \begin{mindmap}
        \begin{mindmapcontent}
          \node (sa) at (current page.center) {Search Algorithms
            \resizebox{\textwidth}{!}{
              \begin{minipage}[t]{8cm}
                \begin{itemize}
                  \item \alert{Notations:}
                  \begin{itemize}
                    \item \alert{Node expansion:} look up adjacent nodes that can be added to the frontier % considering the available incident edges
                      % generating all successor nodes considering the available edges
                    \item \alert{Frontier:} set of all nodes available for expansion (e.g. datastructures like (priority) queue or stack)
                    \item \alert{Search strategy:} defines which node is expanded next
                    \item \alert{Explored set:} set of already visited nodes (e.g. also markings on nodes)
                  \end{itemize}
                \item \alert{Criteria for search strategies:}
                \begin{itemize}
                  \item \alert{Optimality:} If the strategy finds the best path (with the lowest path cost)
                  \item \alert{Completeness:} If the strategy is guaranteed to find a path to a target node when there is one
                  \item \alert{Time complexity} and \alert{Space complexity}
                \end{itemize}
                \end{itemize}
              \end{minipage}
            }
          }
          child {
            node (generalsearch) {General Search
              \resizebox{\textwidth}{!}{
                \begin{minipage}[t]{8cm}
                  \begin{itemize}
                    \item distinction between tree search and graph search is not rooted in the fact whether the problem graph is a tree or a general graph. It is \alert{always assumed} you're dealing with a \alert{general graph}
                    \item The \alert{distinction} lies in the \alert{traversal pattern} that is used to search through the graph, which can be graph-shaped or tree-shaped
                    \begin{itemize}
                      \item If one is dealing with a \alert{tree-shaped problem}, both algorithm variants lead to equivalent results. So one would pick the simpler tree search variant.
                    \end{itemize}
                  \end{itemize}
                \end{minipage}
              }
            }
            child {
              node (ts) {Tree-based search
                \resizebox{\textwidth}{!}{
                  \begin{minipage}[t]{8cm}
                    \begin{itemize}
                      \item nodes are possibly \alert{visited multiple times} (if there are multiple directed paths to a node rooting in the start node), possible leading even to \alert{infinite loops}
                      % \item It will visit a state of the underlying problem graph multiple times, if there are multiple directed paths to it rooting in the start state
                      % \item It is even possible to visit a state an infinite number of times if it lies on a directed loop
                      \begin{itemize}
                        \item but each of this multiple visits would correspond to a different node if one would generate a tree by the nodes visited by the algorithm
                      \end{itemize}
                    \end{itemize}
                  \end{minipage}
                }
                % \resizebox{\textwidth}{!}{
                %   \begin{minipage}[t]{11cm}
                %     \begin{algorithm}[H]
                %       \caption{\pr{Tree-Search}(problem)}
                %       % \begin{pseudo}[kw]
                %       % \fn{initialize} \tn{the} \tt{frontier} \tn{using the initial state of problem}\\
                %       % As long as \tt{frontier} \tn{is not} \cn{empty} do\\+
                %       % \fn{choose} \tn{a} \tt{leaf node} $v$ \tn{and remove it from the} \tt{frontier}\\
                %       %   if \tt{v} \tn{contains a} \cn{goal state} then\\+
                %       %     return \cn{True}\\-
                %       %     \fn{expand} \tn{the} \tt{node}\tn{, adding the resulting nodes to the} \tt{frontier}\\--
                %       % return \cn{False}
                %       % \end{pseudo}
                %       \begin{pseudo}[kw]
                %       \fn{initialize} \tn{the} \tt{frontier} \tn{using the initial state of problem}\\
                %       As long as \tt{frontier} \tn{is not} \cn{empty} do\\+
                %       \fn{choose} \tn{a} \tt{leaf node} \tn{and remove it from the} \tt{frontier}\\
                %         if \tt{node} \tn{contains a} \cn{goal state} then\\+
                %           return \cn{Solution}\\-
                %           \fn{expand} \tn{the} \tt{node}\tn{, adding the resulting nodes to the} \tt{frontier}\\--
                %       return \cn{Failure}
                %       \end{pseudo}
                %     \end{algorithm}
                %     \sourcesone
                %   \end{minipage}
                % }
              }
            }
            child {
              node (gs) {Graph-based search
                \resizebox{\textwidth}{!}{
                  \begin{minipage}[t]{8cm}
                    \begin{itemize}
                      \item Keeps a \alert{explored set} and avoids the problem of tree-based search
                      \item \alert{exponential memory requirements} in the worst case
                    \end{itemize}
                  \end{minipage}
                }
                % \resizebox{\textwidth}{!}{
                %   \begin{minipage}[t]{12cm}
                %     \begin{algorithm}[H]
                %       \caption{\pr{Graph-Search}(problem)}
                %       \begin{pseudo}[kw]
                %       \fn{initialize} \tn{the} \tt{frontier} \tn{using the initial state of problem}\\
                %       \fn{initialize} \tn{the} \tt{explored set} \tn{to be empty}\\
                %       As long as \tt{frontier} \tn{is not} \cn{empty} do\\+
                %       \fn{choose} \tn{a} \tt{leaf node} \tn{and remove it from the} \tt{frontier}\\
                %         if \tt{node} \tn{contains a} \cn{goal state} then\\+
                %           return \cn{True}\\-
                %           \fn{add} \tn{the node to the} \tt{explored set}\\
                %           if \tt{node} \tn{not in the} \tt{frontier} \tn{or} \tt{explored set} then \\+
                %             \fn{expand} \tn{the} \tt{node}\tn{, adding the resulting nodes to the} \tt{frontier}\\---
                %       return \cn{False}
                %       \end{pseudo}
                %     \end{algorithm}
                %   \end{minipage}
                % }
              }
            }
          }
          child {
            node {Uninformed Search Algorithms
              \resizebox{\textwidth}{!}{
                \begin{minipage}[t]{8cm}
                  \begin{itemize}
                    \item  Rigid procedure with no knowledge of the cost of a given node to the goal (e.g. \alert{no heuristic function} $h(v)$), only uses other currently available knowledge (e.g. \alert{path-cost function} $g(v)$ or e.g. \alert{depth function} $d(v)$)
                  \end{itemize}
                \end{minipage}
              }
            }
            child {
              node (ucs) {Uniform-Cost Search
                \resizebox{\textwidth}{!}{
                  \begin{minipage}[t]{10cm}
                    \begin{itemize}
                      \item FIFO datastructure (stack) gets replaced by \alert{priority queue} 
                      \item expands node $v$ from the frontier with \alert{lowest path costs} $g(v)$
                      \item generalization of \alert{Breadth-First Search}, where edge weights can have \alert{different values}
                      \item special case of \alert{Best-First Search}, where $f(v) = g(v)$
                      \item special case of \alert{A$^*$ Search}, where $h(v) = 0$ and thus $f(v) = g(v) + h(v) = g(v)$
                      \begin{itemize}
                        \item if all edge weights would be $1$, then a node would have depth $k$ exactly when it's path costs would be $k$. 
                        \item then the expansion criteria would therefore be the same as for BFS, beacuse a node would have minimal depth exactly when it would have minimum path costs
                      \end{itemize}
                      \item is a modification of \alert{Dijkstra's Algorithm} which is focused on searching a \alert{single shortest path} in terms of \alert{cost} from the \alert{root node} to a \alert{target node} rather than finding the shortest path to every node
                      \begin{itemize}
                        \item it does this by stopping as soon as the target node is found
                        \item it is applicable for both \alert{explicit graphs} and \alert{implicit graphs}, it doesn't need the entire graph as input
                      \end{itemize}
                      \item is \alert{complete}, provided that the branching factor is finite and \alert{optimal}
                      \item \alert{Time complexity:} $O(b^{1+\left\lfloor \frac{C}{\epsilon}\right\rfloor})$
                      \item \alert{Space complexity:} $O(b^{1+\left\lfloor\frac{C}{\epsilon}\right\rfloor})$, priority queue is filled gradually
                    \end{itemize}
                  \end{minipage}
                }
              }
              child {
                node (dijkstra) {Dijkstra's Algorithm
                  \resizebox{\textwidth}{!}{
                    \begin{minipage}[t]{8cm}
                      \begin{itemize}
                        \item determines the \alert{shortest path} in terms of \alert{cost} from the \alert{root node} to \alert{every other node}
                        \item there is \alert{no target node}, processing continues until all nodes have been removed from the priority queue, i.e. until shortest paths to all nodes (not just a target node) have been determined
                        \item is only applicable in \alert{explicit graphs} where the entire graph is given as input
                        \item \alert{Time complexity:} always more time consuming than UCS
                        \item \alert{Space complexity:} adds all nodes to the priority queue at the beginnging with infinite cost
                      \end{itemize}
                    \end{minipage}
                  }
                }
              }
              child {
                node (rbfs) {Breadth-First Search
                  \resizebox{\textwidth}{!}{
                    \begin{minipage}[t]{10cm}
                      \begin{itemize}
                        \item \alert{FIFO} datastructure (\alert{stack})
                        \item expands node $v$ from the frontier with \alert{lowest depth} $d(v)$
                        \item special case of \alert{Uniform-Cost Search} where all edge weights have \alert{no value}
                        \item special case of \alert{Best-First Search}, where $f(v) = d(v)$
                        \item \alert{complete}, provided that the branching factor is finite and \alert{optimal}, provided every edge has identical, non-negative weights
                        \item \alert{Time complexity:} $O({\mid} V{\mid}+{\mid} E{\mid})$, $b + b^2 + \ldots + b^d\in O(b^d)$ is the maximal number of nodes expanded
                        \begin{itemize}
                          \item If the algorithm were to apply the goal test to nodes when selected for expansion rather than when generated, the whole layer of nodes at depth $d$ would be expanded before the goal was detected and the time complexity would be $O(b^{d+1})$
                        \end{itemize}
                        \item \alert{Space complexity:}
                        \begin{itemize}
                          \item \alert{tree based:} $O({\mid} V{\mid}) = O(b^d)$ for the frontier%, every explored node is kept in memory
                          \item \alert{graph based:} $O(b^d)$ for the frontier and $O(b^{d-1})$ for the explored set
                        \end{itemize}
                        % \item \alert{Time complexity:} the maximal number of nodes expanded is $b+b^{2}+b^{3}+\ldots +b^{d}\in O(b^{d})$
                        % \item \alert{Space complexity:} Every node generated is \alert{kept in memory}. Therefore space needed for the \alert{frontier} is $O(b^d)$ and for the \alert{explored set} $O(b^{d-1})$
                      \end{itemize}
                    \end{minipage}
                  }
                  \resizebox{\textwidth}{!}{
                    \begin{minipage}[t]{10.5cm}
                      \bfs
                    \end{minipage}
                  }
                }
                child {
                  node (bidirectional) {Bidirectional Search}
                }
              }
            }
            child {
              node (dfs) {Depth-First Search
                \resizebox{\textwidth}{!}{
                  \begin{minipage}[t]{10cm}
                    \begin{itemize}
                      \item \alert{LIFO} datastracture (LIFO-\alert{queue})
                      \item expands node $v$ from the frontier with \alert{greatest depth} $d(v)$
                      \item special case of \alert{Best-First Search}, where $f(v) = -d(v)$
                      \item in general, the path found is \alert{not optimal} and \alert{completeness} can be guaranteed \alert{only for graph-based search} and \alert{finite graphs}
                      \item \alert{Time complexity:}
                        \begin{itemize}
                          \item \alert{graph-based:} $O({\mid} V{\mid}+{\mid} E{\mid})$, search bounded by the number of nodes (might be infinite)
                          \item \alert{tree-based:} algorithm might generate $O(b^m)$ nodes in the search tree which might be much larger than the number of nodes
                            %   \begin{itemize}
                            %     \item $m$ is the maximum length of a path in the state space
                            %   \end{itemize}
                        \end{itemize}
                        \item \alert{Space complexity:} 
                          \begin{itemize}
                            \item \alert{graph-based:} $O({\mid} V{\mid})$, in worst case, all nodes need to be stored in the explored set (no advantage over breadth-first)
                            \item \alert{tree-based:} $O(b\cdot m)$, needs to store only the nodes along the path from the root to the leaf node. Once a node has been expanded, it can be removed from memory as soon as all its descendants have been fully explored
                          \end{itemize}
                      \end{itemize}
                  \end{minipage}
                }
                \resizebox{\textwidth}{!}{
                  \begin{minipage}[t]{11cm}
                    \dfs
                  \end{minipage}
                }
              }
              child {
                node {Recursive Depth-First Search
                  \resizebox{\textwidth}{!}{
                    \begin{minipage}[t]{11cm}
                      \dfsrec
                    \end{minipage}
                  }
                }
              }
              child {
                node {Depth-Limited Search}
                child {
                  node {Iterative-Deepening Search}
                }
              }
            }
          }
          child {
            node {Informed Search Algorithms
              \resizebox{\textwidth}{!}{
                \begin{minipage}[t]{8cm}
                  \begin{itemize}
                    \item Knowledge of the worth of expanding a node $v$ is given in the form of an \alert{evaluation function} $f(v)$, which assigns a real number to each node
                    \item Often $f(v)$ includes a \alert{heuristic function} $h(v)$ as a component, which estimates the costs of the cheapest path from $v$ to the goal
                  \end{itemize}
                \end{minipage}
              }
            }
            child {
            node {Local Search Algorithms
              \resizebox{\textwidth}{!}{
                \begin{minipage}[t]{8cm}
                  \begin{itemize}
                    \item if it is unimportant how the goal is reached, \alert{only the goal itself matters} and if in addition a \alert{quality} measure for nodes is given
                    \item it operates using a \alert{single current node} (rather than multiple paths)
                    \item requires little memory
                  \end{itemize}
                \end{minipage}
              }
            }
            child {
              node {Hill Climbing
                \resizebox{\textwidth}{!}{
                  \begin{minipage}[t]{8cm}
                    \begin{itemize}
                      \item Begin with a randomly-chosen configuration and \alert{improve} on it \alert{step by step}
                    \end{itemize}
                  \end{minipage}
                }
              }
              child {
                node {Simulated Annealing
                  \resizebox{\textwidth}{!}{
                    \begin{minipage}[t]{8cm}
                      \begin{itemize}
                        \item \enquote{noise} is injected systematically, first a lot, then gradually less
                      \end{itemize}
                    \end{minipage}
                  }
                }
              }
              child {
                node {Gradient Descent}
              }
            }
          }
          child {
            node {Genetic Algorithms
              \resizebox{\textwidth}{!}{
                \begin{minipage}[t]{8cm}
                  \begin{itemize}
                    \item Similar to \alert{evolution}, we search for solutions by three operators: \alert{mutation}, \alert{crossover}, and \alert{selection}
                  \end{itemize}
                \end{minipage}
              }
            }
          }
            child {
              node (bfs) {Best-First Search 
                \resizebox{\textwidth}{!}{
                  \begin{minipage}[t]{8cm}
                    \begin{itemize}
                      \item informed search procedure that expands the node with the \enquote{\alert{best}} $f$-value first
                      \item is an instance of the general \alert{Tree-Search} algorithm in which frontier is a \alert{priority queue} ordered by an \alert{evaluation function} $f$
                      \begin{itemize}
                        \item when $f$ is always correct, ones does not need to search
                      \end{itemize}
                    \end{itemize}
                  \end{minipage}
                }
              }
              child {
                node {Greedy Search
                  \resizebox{\textwidth}{!}{
                    \begin{minipage}[t]{8cm}
                      \begin{itemize}
                        \item A \alert{best-first} search using the \alert{heuristic function} $h(v)$ as the \alert{evaluation function}, i.e. $f(v) = h(v)$ is called a \alert{greedy search}
                        \begin{itemize}
                          % \item judge the \enquote{worth} of a node by estimating its path costs to the target node
                          \item $h(v) =$ estimated path-costs from $v$ to the target node
                          \item the only real restriction is that $h(v) = 0$ if $v$ is the target node
                        \end{itemize}
                        \item is generally \alert{incomplete} and \alert{not optimal}
                        \begin{itemize}
                          \item \alert{graph-search} version is complete only in finite graphs
                        \end{itemize}
                      \end{itemize}
                    \end{minipage}
                  }
                }
                child {
                  node (hf) {Heuristic function
                    \resizebox{\textwidth}{!}{
                      \begin{minipage}[t]{8cm}
                        \begin{itemize}
                          \item or simply a \alert{heuristic}
                          \item the evaluation function $f$ in \alert{greedy searches} is a heuristic function $h$ 
                          \item the heuristic is \alert{problem-specific} and \alert{focuses} the search
                          \item \alert{In AI it has two meanings:}
                          \begin{itemize}
                            \item Heuristics are \alert{fast} but in certain situations \alert{incomplete} methods for problem-solving
                            \item Heuristics are methods that \alert{improve the search} in the \alert{average-case}
                          \end{itemize}
                        \item The word heuristic is derived from the Greek word {\gyre ευρισκειν} (note also: {\gyre ευρηκα!})
                        \end{itemize}
                      \end{minipage}
                    }
                  }
                }
              }
              child {
                node (astar) {A$^*$
                  \resizebox{\textwidth}{!}{
                    \begin{minipage}[t]{8cm}
                      \begin{itemize}
                        \item A$^*$ combines \alert{greedy search} with the \alert{uniform-cost search}
                        \item Always expands node with lowest \alert{evaluation function} $f(v)$ first, where:
                        \begin{itemize}
                          \item $g(v) =$ actual cost from the start node to $v$
                          \item $h(v) =$ estimated cost from $v$ to the nearest target node
                          \item $f(v) = g(v) + h(v) =$ the estimated cost of the cheapest path through $v$
                        \end{itemize}
                        \item We require that for A$^*$, that $h$ is admissible (e.g. straight-line distance is admissible)
                        \begin{itemize}
                          \item $h$ is an \alert{optimistic estimate} of the costs that actually occur
                        \end{itemize}
                        \item \alert{complete}, provided that every node has a finite number of successor nodes and there exists a positive constant $\delta > 0$ such that every edge has at least weight $\delta$ and \alert{optimal}, provided that one uses the \alert{tree-based} variant
                        \begin{itemize}
                          \item for the \alert{graph-based} variant, one:
                          \begin{itemize}
                            \item either needs to consider re-opening nodes from the explored set, when a better estimate becomes known, or
                            \item one needs needs to require stronger restrictions on the heuristic estimate: it needs to be \alert{consistent}
                            \item $A^*$ can still be applied if heuristic is not consistent, but \alert{optimality is lost} in this case
                          \end{itemize}
                        \end{itemize}
                        \item \alert{Time complexity:} $O(b^d)$, exponential in the path length of the solution. More refined complexity results depend on the assumptions made, e.g. on the quality of the heuristic function
                        \item \alert{Space complexity:} $O(b^d)$, exponential in the path length of the solution. Roughly the same as that of all other graph search algorithms, as it keeps all generated nodes in memory
                      \end{itemize}
                    \end{minipage}
                  }
                }
                child {
                  node {Iterative-Deepening A$^*$ (IDA$^*$)
                %     \resizebox{\textwidth}{!}{
                %       \begin{minipage}[t]{8cm}
                %         \begin{itemize}
                %           \item the $f$-costs are used to define the cutoff (rather than the depth of the search tree)
                %         \end{itemize}
                %       \end{minipage}
                %     }
                  }
                }
                child {
                  node {Recursive Best First Search (RBFS)}
                }
                child {
                  node {Memory-bounded A$^*$ (MA$^*$) and Simplified MA$^*$ (SMA$^*$)}
                }
                child {
                  node {Admissible
                    \resizebox{\textwidth}{!}{
                      \begin{minipage}[t]{8cm}
                        \begin{itemize}
                          \item $h$ is admissible if the following holds for all $v$: $h(v) \le h^*(v)$
                          \begin{itemize}
                            \item $h^*(v)$ are the actual cost of the optimal path from $v$ to the target node
                          \end{itemize}
                        \end{itemize}
                      \end{minipage}
                    }
                  }
                }
                child {
                  node {Consistent
                    \resizebox{\textwidth}{!}{
                      \begin{minipage}[t]{8cm}
                        \begin{itemize}
                          \item A heuristic $h$ is called consistent \alert{iff} for all edges $v$ leading from $s$ to $s'\colon h(s) - h(s') \le c(v)$, where $c(v)$ denotes the weight of edge $v$
                          \item Consistent heuristics prevent the need to re-open nodes from the explored set
                          \item Consistency implies \alert{admissibility}
                        \end{itemize}
                      \end{minipage}
                    }
                  }
                }
              }
            }
          };
        \end{mindmapcontent}
        \begin{edges}
          \edge{dfs}{bidirectional}
          \edge{ts}{bfs}
          % \edge{ts}{rbfs}
          % \edge{gs}{dfs}
          \edge{astar}{ucs}
          \edge{bfs}{ucs}
          % \edge{bfs}{rbfs}
          \edge{bfs}{dfs}
          \edge{hf}{astar}
        \end{edges}
        \annotation{dfs}{\bd}
        \annotation{rbfs}{\bd}
        \annotation{generalsearch}{\sourcesone}
        \annotation{dijkstra}{
          \resizebox{\textwidth}{!}{
            \begin{minipage}[t]{4cm}
              \cite{vineetnayak28AnswerWhatDifference2016}
            \end{minipage}
          }
        }
      \end{mindmap}
    \end{minipage}
  }
\end{frame}

% vielleicht noch was mit reachability problem lösen
