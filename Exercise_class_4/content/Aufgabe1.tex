%!Tex Root = ../main.tex
% ./Packete.tex
% ./Design.tex
% ./Deklarationen.tex
% ./Vorbereitung.tex
% ./Aufgabe2.tex
% ./Aufgabe3.tex
% ./Aufgabe4.tex
% ./Appendix.tex

\section{Exercise 1}

\setcounter{exercise}{1}

\begin{frame}[allowframebreaks]{Aufgabe \thesection}{Maximum Cut, Maximal Cut, Perfect Cut, Vertex Cover}
  \begin{solution}
    \ctikzfig{1a}
  \end{solution}
  \begin{solution}
    \ctikzfig{1a_sol}
  \end{solution}
  \begin{solution}
    \ctikzfig{1b}
  \end{solution}
  \begin{solutionnoinc}
    \begin{columns}
      \begin{column}{0.5\textwidth}
        \ctikzfig{1b_sol}
      \end{column}
      \begin{column}{0.5\textwidth}
        \ctikzfig{1c}
      \end{column}
    \end{columns}
  \end{solutionnoinc}
  \begin{solution}
    \ctikzfig{1c_diff}
  \end{solution}
  \begin{solution}
    \begin{columns}
      \begin{column}{0.5\textwidth}
        \begin{itemize}
          \item maximum matching $\alpha'(G) = 6$
          \item minimum vertex cover $\beta(G) = 6$
          \begin{itemize}
            \item \alert{König und Egerváry:} For any bipartite graph $G$ the size of maximum matching equals the size of minimum vertex cover
          \end{itemize}
          \item maximum independant set $\alpha(G) = 12 - 6 = 6$ 
          \begin{itemize}
            \item $\alpha(G) + \beta(G) = n(G)$
            % \item \alert{dual problem with edge cover:} The size of a (minimal) edge cover is at least the size of a maximal independed set
          \end{itemize}
          \item minimum edge cover $\beta'(G) = 12 - 6 = 6$
          \begin{itemize}
            \item \alert{Gallai:} $\alpha'(G) + \beta'(G) = n(G)$
          \end{itemize}
        \end{itemize}
      \end{column}
      \begin{column}{0.5\textwidth}
        \ctikzfig{1e}
      \end{column}
    \end{columns}
  \end{solution}
  \begin{solutionnoinc}
    \ctikzfig{1f}
  \end{solutionnoinc}
  \begin{solution}
    \ctikzfig{1f2}
  \end{solution}
\end{frame}
